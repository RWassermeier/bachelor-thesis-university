\documentclass[
a4paper, %DinA4-Format
oneside, %einseitig, für Bücher: twoside
pdftex, %für pdflatex
12pt, %Schriftgröße im gesamten Dokument, Basisschriftgröße
headsepline, %Linie zwischen Kopfzeile und Text
footsepline, %Linie zwischen Text und Fußzeile
bibtotoc, %Bibliographie im Inhaltsverzeichnis
idxtotoc, %Index im Inhaltsverzeichnis
headings=twolinechapter % Kapitel zweizeilig, nur für Anhang: appendixprefix=true, in beiden Fällen entfällt das Anhang-Konstrukt
]{scrbook}

%Seiteneinstellungen
\usepackage[utf8]{inputenc} %unicode-codierung
\usepackage[T1]{fontenc} % Zeichencodierung der verwendeten Schrift
\usepackage[ngerman]{babel} %Sprachauswahl, n für neue Rechtschreibung
\usepackage{setspace} % Für Zeilenabstand

%Seitenränder
\usepackage[
inner=3cm,
outer=3cm,
top=2cm,
bottom=2cm,
includeheadfoot]
{geometry}
\usepackage{scrtime}
\usepackage{ifthen}
% Entwurfsmodus
%
\KOMAoption{draft}{false}                                                       % Entwurfsmodus?
\def\printversion{false}                                                        % Printversion (alle Hyperlinks schwarz-weiß)?
\def\versionlabel{true}      
\usepackage{scrtime}
\usepackage{ifthen}                                                   % Versionsangabe im footer?
\newboolean{printversion}
\setboolean{printversion}{\printversion}
\newcommand{\finalVersionString}{}
\ifthenelse{\boolean{\versionlabel}}{%
  \usepackage[draft]{prelim2e}
      \renewcommand{\PrelimWords}{\relax}
      \renewcommand{\PrelimText}{%
        \color{red}\tiny[\,Version: \today, \thistime\ Uhr\,]%
      }
      \renewcommand{\finalVersionString}{Entwurf, \today}
}{
  \renewcommand{\finalVersionString}{Eingereichte Version, \today}
}
%Kopf- und Fußzeilen
\usepackage{scrpage2}
\pagestyle{scrheadings} %Seitenstil wird auf scrheadings gesetzt
\clearscrheadfoot
\ihead[]{} %innere bzw. linke Kopfzeileneinträge
\chead[]{\headmark} %mittlere Kofzeileneinträge mit Kapitelüberschrift
\ohead[]{} %äußere bzw. rechte Kopzeileneinträge
\ifoot[]{}
\cfoot[]{\pagemark} %Seitenzahlen mittig
\ofoot[]{}
\renewcommand*{\chapterpagestyle}{scrheadings} % Überschrift in Kopfzeile ab erster Seite

% Schriftauswahl folgend
\usepackage{mathptmx}
\usepackage[scaled=0.92]{helvet}
\usepackage{courier}

% math. und phys. Formeln
\usepackage{amsmath}
\usepackage{amsfonts}
\usepackage{amssymb}
\usepackage{amsthm}
\usepackage{upgreek} %griech. Buchstaben in Formelumgebung nicht kursiv
\usepackage[squaren]{SIunits} % Für SI-units [squaren] ist obsolet, muss aber sein

\usepackage[babel,german=quotes]{csquotes} %für Zitate
\usepackage[
backend=biber,
sorting=nty, %Sortierung im Literaturverzeichnis nach Name, Titel, Jahr
   style=numeric,
    sortlocale=de_DE,
    natbib=true,
    url=true,
    doi=true,
    eprint=false
]{biblatex} %dieses usepackage muss weg, wenn ich nach DIN-Norm1505 zitiere, siehe Verzeichnis
\bibliography{literatur} %das muss weg, wenn ich nach DIN-Norm1505 zitiere, siehe Verzeichnis
%\usepackage{cite} %nur wenn biblatex auskommentiert ist, dann das nicht auskommentieren

% definiert \FootCite, dass sorgt im Gegensatz zu \footcite dafür, dass in der Fussnote [<cite>]. statt <cite>. gesetzt wird

\usepackage{xparse}

\NewDocumentCommand\FootCite{O{}O{}m}{%
	{\footnote{\cite[#1][#2]{#3}.}}
}

% FootCite kann wie folgt genutzt werden:
% 1 Parameter: \FootCite{referenz}, wobei referenz der Literaturverweis sein muss
% pre- und post:
% erlaubt:
% \FootCite[vgl.][]{referenz} => [vgl. referenz]
% \FootCite[][S. 88]{referenz} => [referenz, S. 88]
% \FootCite[vgl.][S. 88]{referenz} => [vgl. referenz, S. 88]
% verboten:
% \FootCite[S. 88]{referenz}


% Tabellen
\usepackage{array}
\usepackage{longtable}
\newcolumntype{C}[1]{>{\centering\arraybackslash}p{#1}} % use: C{<breite>}
\newcolumntype{R}[1]{>{\raggedleft\arraybackslash}p{#1}} % use: R{<breite>}
\usepackage{hhline}
\usepackage{multicol} % Fasst Spalten einer Tabelle zusammen
\usepackage{multirow} % Fasst Zeilen einer Tabelle zusammen
\usepackage{rotating} % Um Objekte zu drehen

% Graphiken
\usepackage[table,dvipsnames]{xcolor}
\usepackage{graphicx}
\usepackage{epstopdf}
\usepackage{subfigure}
\graphicspath{{Bilder/}}
\usepackage{float} % Notwendig, um floating abzuschalten

\usepackage[novbox]{pdfsync} % Für Synchronisation tex - pdf

\usepackage{Befehle}

% Sonstiges
\usepackage{enumerate} % Definiert enumerate-Umgebung, z.B.: \begin{enumerate}[a)] \end{enumerate}

\usepackage{bookmark}

% immer als letztes Paket
\usepackage{hyperref}
\hypersetup{
colorlinks=true,
linkcolor=black,
citecolor=black,
urlcolor=black}

% neudefinierte Mathe-Operatoren
\DeclareMathOperator{\e}{e}
\DeclareMathOperator{\Div}{div}
\DeclareMathOperator{\Rot}{rot}
\DeclareMathOperator{\Grad}{grad}
\DeclareMathOperator{\ld}{ld}
% Deutsche Bezeichner für \autoref{}
\addto\extrasngerman{%
  \def\subsectionautorefname{Abschnitt}%
  \def\definitionautorefname{Definition}%
  \def\algorithmautorefname{Algorithmus}%
  \def\subfigureautorefname{Abbildung}%
  \def\tableautorefname{Tabelle}%
}

\graphicspath{{Bilder/}}  
%
% neudefinierte Befehle
\newcommand{\Abb}[1]{Abbildung \ref{#1}}
\newcommand{\Sei}[1]{Seite \pageref{#1}}
\newcommand{\Tab}[1]{Tabelle \ref{#1}}
\newcommand{\gl}[1]{Gleichung \ref{#1}}
\newcommand{\Anm}[1]{{\color{red}#1}}